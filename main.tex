%%%%%%%%%%%%%%%%%%%%%%%%%%%%%%%%%
% Cerca la parola TODO all'interno di questo file per individuare ciò che devi personalizzare.
%%%%%%%%%%%%%%%%%%%%%%%%%%%%%%%%%
%
%
%


% Comandi per generare PDF-A2 con i relativi metadati 
\begin{filecontents*}[overwrite]{\jobname.xmpdata}
\Title{TODO Titolo del documento PDF}
\Author{TODO Nome autore}
\Language{it}
\Subject{TODO Breve frase descrittiva}
\Keywords{TODO\sep keyword2\sep keyword3}
\end{filecontents*}


%%%%%%%%%%%%%%%%%%%%%%%%%%%%%%%%%
%   PREAMBOLO DEL DOCUMENTO     %
%%%%%%%%%%%%%%%%%%%%%%%%%%%%%%%%%
\documentclass[a4paper,12pt,oneside,top=3cm,bottom=3cm,left=3.5cm,right=3.5cm,openright,reqno,table]{book}
% openany - fa iniziare i capitoli direttamente nella pagina successiva
% openright - fa iniziare i capitoli nella prima pagina destra disponibile 
% fleqn  - allinea le formule a sinistra anzichè centrarle
% leqno - dispone la numerazione delle formule sulla sinistra o destra
% reqno - dispone la numerazione delle formule sulla destra

% package fondamentali
\usepackage[T1]{fontenc}
\usepackage[utf8]{inputenc}
% se si realizza un documento in piu lingue bisogna istruire babel con
% [lingua_secondaria,lingua_principale]
\usepackage[italian]{babel}
\usepackage{colorprofiles}
\usepackage[colorlinks,hyperindex,pagebackref]{hyperref}
\hypersetup{
			citecolor=black,
			filecolor=black,
			linkcolor=black,
			urlcolor=black
}
% per generare il PDF-A2
\usepackage[a-2b,mathxmp]{pdfx}[2018/12/22]
\hypersetup{pdfstartview=}

% tutti i package sono specificati in un file a parte: packages.sty
\usepackage{packages}

% comandi personalizzati
\usepackage{utils}

% stile dei capitoli
\usepackage[SeSa]{fncychap}
% gli stili di capitolo disponibili sono i seguenti:
% SeSa, Sonny, Lenny, Rejne, Conny, PetersLenny, Bjornstrup, Glenn, Bjarne, Nicola

\linespread{1.5}

%%%%%%%%%%%%%%%%%%%%%%%%%%%%%%%%%
%   DOCUMENTO VERO E PROPRIO    %
%%%%%%%%%%%%%%%%%%%%%%%%%%%%%%%%%
\begin{document}

% frontespizio 
\begin{titlepage}
\changepage{}{}{}{-7.5 mm}{}{}{}{}{}
% parametri per cambiare le dimensioni di una singola pagina in ordine:
% {textheight}{textwidth}{evensidemargin}{oddsidemargin}{columnsep}
% {topmargin}{headheight}{headsep}{footskip}
% se voglio centrare la pagina devo mettere bindingoffset/2
% i primi 5 parametri posso usarli con \changetext


\begin{center}
\includegraphics [width=.15\columnwidth, angle=0]{logo-unisa}\\ % height
\vspace{0.5cm}
{\Large \scshape Università degli Studi di Salerno}\\
\vspace{0.5cm}
{\Large Dipartimento di Informatica}\\
\vspace{0.5cm}
{\Large Corso di Laurea Triennale in Informatica}\\
\vspace{1.5cm}
{\Large \scshape Tesi di Laurea} \\
\vspace{4cm}
{\Huge \bfseries Titolo della Tesi} \\
\vspace{4cm}

\begin{minipage}[t]{7cm}
\flushleft
{\large \textsc{Relatore}}

{\large Prof. Nome relatore} \\
{\large Dott. Nome tutor} \\
Università degli Studi di Salerno \\[0.25cm]
\end{minipage}
\hfill
\begin{minipage}[t]{7cm}
\flushright
{\large \textsc{Candidato}}

{\large \textbf{Nome Cognome}} \\
Matricola: 0123456789
\end{minipage}

\vspace{3cm}

{\small Anno Accademico YYYY-YYYY} %\\
%
%
\end{center}

\end{titlepage}


% sesalab
\begin{titlepage}
\nonumber
\null \vspace {\stretch{1}}
\begin{flushright}
{\textit{Questa tesi è stata realizzata nel} \hspace*{0.25cm}}
\\
\vspace{0.5cm}
\includegraphics[width=.40\columnwidth, angle=0]{logo-sesa}
\end{flushright}
\end{titlepage}


%dedica
\begin{titlepage}
\nonumber
\null \vspace {\stretch{1}}
	\begin{flushright}
%	\begin{verse}
\textit{Dedica o citazione} \\[5mm]
%	\end{verse}
	\end{flushright}
\end{titlepage}

% abstract
\renewcommand{\abstractname}{Abstract}
\begin{titlepage}
\begin{abstract}
%
Il testo dell'abstract qui...
%
\\[1cm]
\end{abstract}
\end{titlepage}

\frontmatter
% quello che segue è in numerazione romana e i capitoli non verranno numerati
% se non si vuole che compaia il numero di pagina basta usare il comando:
%\nonumber

% indici 
\phantomsection

% Il simbolo * serve per evitare che comapaia nell'indice

\tableofcontents
\clearpage

\addcontentsline{toc}{chapter}{Elenco delle Figure}
\listoffigures
\clearpage

\addcontentsline{toc}{chapter}{Elenco delle Tabelle}
\listoftables
\clearpage



\mainmatter
% quello che segue sarà in numerazione araba e i capitoli verranno numerati

% TODO inserire capitoli
\chapter{Introduzione}

\begin{comment}
Ciò che viene scritto in questo enviroment non sarà mostrato nel pdf.
Questo comando è utile per nascondere un intero paragrafo senza eliminarlo dal file sorgente. 
\end{comment}

Lorem ipsum dolor sit amet, consectetur adipiscing elit, sed do eiusmod tempor incididunt ut labore et dolore magna aliqua. Velit sed ullamcorper morbi tincidunt ornare massa. Lacus luctus accumsan tortor posuere ac. Cursus mattis molestie a iaculis at erat pellentesque. Ipsum dolor sit amet consectetur adipiscing elit. Fermentum iaculis eu non diam phasellus vestibulum lorem sed risus. Massa sapien faucibus et molestie ac feugiat sed. Faucibus turpis in eu mi. Ipsum dolor sit amet consectetur adipiscing elit ut aliquam. Quis eleifend quam adipiscing vitae proin sagittis nisl. Sollicitudin tempor id eu nisl nunc mi. Nulla facilisi etiam dignissim diam quis enim lobortis scelerisque fermentum.

Vel elit scelerisque mauris pellentesque pulvinar pellentesque habitant morbi. Duis at consectetur lorem donec massa sapien faucibus. Mattis pellentesque id nibh tortor. Ut sem nulla pharetra diam sit. Arcu felis bibendum ut tristique et egestas quis ipsum suspendisse. Ac tortor dignissim convallis aenean et. Elit at imperdiet dui accumsan sit amet nulla. Libero id faucibus nisl tincidunt eget nullam non. Dictumst quisque sagittis purus sit amet volutpat consequat. Lectus arcu bibendum at varius vel pharetra vel turpis nunc. Enim tortor at auctor urna nunc id cursus metus aliquam. Urna nec tincidunt praesent semper feugiat nibh sed pulvinar proin. Sed elementum tempus egestas sed sed. Ullamcorper velit sed ullamcorper morbi tincidunt ornare massa. Id nibh tortor id aliquet lectus proin. Diam sollicitudin tempor id eu nisl. Tincidunt eget nullam non nisi est sit amet. Sed arcu non odio euismod lacinia at quis risus sed. Ultrices gravida dictum fusce ut placerat orci nulla. Sed id semper risus in hendrerit.

Viverra nibh cras pulvinar mattis nunc sed blandit libero volutpat. Sit amet nisl suscipit adipiscing bibendum. Sit amet porttitor eget dolor morbi non arcu risus. Aliquam eleifend mi in nulla posuere sollicitudin aliquam. Ornare suspendisse sed nisi lacus sed. Ante in nibh mauris cursus mattis molestie a iaculis at. Nec ultrices dui sapien eget mi proin sed libero. Fringilla ut morbi tincidunt augue interdum. Pharetra massa massa ultricies mi quis hendrerit. Phasellus faucibus scelerisque eleifend donec pretium vulputate sapien nec. Mauris commodo quis imperdiet massa tincidunt nunc pulvinar sapien. Dui nunc mattis enim ut tellus elementum sagittis vitae. Pulvinar elementum integer enim neque volutpat. Non tellus orci ac auctor augue mauris augue neque. Vitae ultricies leo integer malesuada nunc vel risus commodo. Et tortor at risus viverra adipiscing at in tellus. Est ullamcorper eget nulla facilisi etiam. Rhoncus est pellentesque elit ullamcorper dignissim cras. Est velit egestas dui id ornare. Elit ullamcorper dignissim cras tincidunt lobortis feugiat vivamus at.

Dolor morbi non arcu risus quis varius. Cursus in hac habitasse platea dictumst quisque. Amet nulla facilisi morbi tempus iaculis urna id volutpat lacus. Id nibh tortor id aliquet. Ac turpis egestas maecenas pharetra convallis posuere. Platea dictumst vestibulum rhoncus est pellentesque elit ullamcorper dignissim. Quis varius quam quisque id diam vel quam elementum. Adipiscing elit pellentesque habitant morbi tristique. Commodo quis imperdiet massa tincidunt nunc pulvinar. Eget nunc scelerisque viverra mauris in aliquam sem fringilla. Mauris a diam maecenas sed enim ut sem.

\chapter{Esempio Capitolo} %\label{1cap:spinta_laterale}
% [titolo ridotto se non ci dovesse stare] {titolo completo}
%

%\begin{citazione}
%BREVE SPIEGAZIONE CONTENUTO CAPITOLO
%\end{citazione}

\section{Prima sezione}

\subsection{Prima sottosezione}

\paragraph{Primo paragrafo} Questa è una prova di un testo di un paragrafo.

The well known Pythagorean theorem $x^2 + y^2 = z^2$\footnote{\url{http://google.com}} was 
proved to be invalid \textbf{\textit{for other exponents}}. 

\textsc{com.java.xxx}

``Lorem ipsum''

Meaning the next equation has no integer solutions:

\[ x^n_m + y^n = z^n \]

\begin{figure}[h]
\includegraphics[width=3cm]{figure/picture.pdf}
\centering
\caption{Girasole piantato nel cortile del Dipartimento di Informatica.}
\end{figure}






\begin{table}
    \centering
    \caption{Questions to decide whether to conduct a MLR~\cite{garousi2019_mlr_guidelines}.}
    \vspace{1mm} % Adjust the height of the space between caption and tabular
    
    \rowcolors{1}{graytable}{white}
    \resizebox{\linewidth}{!}{
    \begin{tabular}{p{0.025\linewidth} p{0.875\linewidth} P{0.1\linewidth}}
        \toprule
        \rowcolor{black}
        \textbf{\textcolor{white}{\#}} & \textbf{\textcolor{white}{Question}} & \textbf{\textcolor{white}{Answer}}\\
        \bottomrule
        1 & Is the subject ``complex'' and not solvable by considering only the formal literature? & Yes\\
        2 & Is there a lack of volume or quality of evidence, or a lack of consensus of outcome measurement in the formal literature? & No\\
        3 & Is the contextual information important to the subject under study? & Yes\\
        4 & Is it the goal to validate or corroborate scientific outcomes with practical experiences? & Yes\\
        5 & Is it the goal to challenge assumptions or falsify results from practice using academic research or vice versa? & Yes\\
        6 & Would a synthesis of insights and evidence from the industrial and academic community be useful to one or even both communities? & Yes\\
        7 & Is there a large volume of practitioner sources indicating high practitioner interest in a topic? & Yes\\
        \bottomrule
        \rowcolor{white}
        \multicolumn{3}{p{\linewidth}}{\textit{\textbf{Note}: The possible answers to each question are ``Yes'' or ``No''. One or more ``Yes'' responses suggest that it could be useful to conduct a MLR.}}\\
    \end{tabular}
    }
    
    \label{table:motivations_MLR}
\end{table}









\begin{center}
\begin{table}[!h]
\begin{tabular}{||c c c ||}
 \hline
 Col1 & Col2 & Col3 \\ [0.5ex] 
 \hline\hline
 1 & 6 & 787 \\ 
 \hline
 2 & 7 & 5415 \\
 \hline
 3 & 545 & 7507 \\
 \hline
 4 & 545 & 7560 \\
 \hline
 5 & 88 & 6344 \\ [1ex] 
 \hline
\end{tabular}
\caption{Esempio di tabella}
\end{table}
\end{center}

The well known Pythagorean theorem $x^2 + y^2 = z^2$ was 
proved to be invalid for other exponents. 
Meaning the next equation has no integer solutions:

\[ x^n_m + y^n = z^n \]

\begin{minted}
[
frame=lines,
framesep=2mm,
baselinestretch=1.2,
bgcolor=LightGray,
fontsize=\footnotesize,
linenos
]
{python}
import numpy as np
    
def incmatrix(genl1,genl2):
    m = len(genl1)
    n = len(genl2)
    M = None #to become the incidence matrix
    VT = np.zeros((n*m,1), int)  #dummy variable
    
    #compute the bitwise xor matrix
    M1 = bitxormatrix(genl1)
    M2 = np.triu(bitxormatrix(genl2),1) 

    for i in range(m-1):
        for j in range(i+1, m):
            [r,c] = np.where(M2 == M1[i,j])
            for k in range(len(r)):
                VT[(i)*n + r[k]] = 1;
                VT[(i)*n + c[k]] = 1;
                VT[(j)*n + r[k]] = 1;
                VT[(j)*n + c[k]] = 1;
                
                if M is None:
                    M = np.copy(VT)
                else:
                    M = np.concatenate((M, VT), 1)
                
                VT = np.zeros((n*m,1), int)
    
    return M
\end{minted}

Questo è un elenco puntato
\begin{itemize}
    \item primo elemento
    \begin{itemize}
    \item primo sottoelemento
    \end{itemize}
    \item secondo elemento
    \item terzo elemento
\end{itemize}

Questo è un elenco numerato
\begin{enumerate}
    \item primo elemento
    \begin{enumerate}
    \item primo sottoelemento
    \end{enumerate}
    \item secondo elemento
    \item terzo elemento
\end{enumerate}

Questa è una frase che dice qualcosa, e deve essere supportata da una citazione.
Se gli autori sono più di due, scriviamo che Garousi et al. hanno detto qualcosa~\cite{garousi2019_mlr_guidelines}.
Se gli autori sono due, scriviamo che Bruegge e Dutoit hanno detto cose~\cite{bruegge2009object_se}.
Quest'altra frase dice qualcos'altro, ma non so in quale paper l'ho letta~\needcite.
Questa frase dice cose supportate da diverse pubblicazioni~\cite{casillo2022detecting,destefano2020splicing,giordano2022slr,iannone2022secret,lambiase2022fences,sellitto2022refactoring,pontillo2022static}.
In questa parte devo ricordarmi qualcosa\nb{devo ricordarmi di questa cosa}.

\revised{Questa parte del documento è stata modificata dopo che il relatore mi ha dato suggerimenti.}

Invece qui ho avuto dei feedback dal relatore, e devo ricordarmi di applicare i commenti. \feedback{aggiungi un paragrafo sulle research question}

\goal{Questo è l'obiettivo della tesi.}

\rqbox{1}{Quale è la nostra prima Research Question?}

\rqanswer{1}{Questa è la risposta alla prima RQ.}

\finding{1}{Questo è il primo finding che riportiamo tra i risultati.}

\takeaway{1}{Primo takeaway message.}


\chapter{Conclusioni} 

Le conclusioni qui...



\backmatter

% bibliografia in stile IEEE
\addcontentsline{toc}{chapter}{\bibname}
\bibliographystyle{IEEEtran}
\bibliography{references}

% ringraziamenti
\chapter*{Ringraziamenti}
\pagenumbering{gobble}
\fancyhf{}

Ringraziamenti qui...


% se hai intenzione di donare un albero tramite treedom
%\begin{titlepage}
\nonumber
\null \vspace {\stretch{1}}
\begin{flushright}
{\textit{Questa tesi ha contribuito a piantare un albero in Kenya tramite il progetto Treedom.}}
\\
\vspace{0.5cm}
{\footnotesize\url{https://www.treedom.net/it/user/sesalab/event/sesa-random-forest}}
\end{flushright}
\end{titlepage}


\end{document}

%%%%%%%%%%%%%%%%%%%%%%%%%%%%%%%%%