\chapter{Risultati} %\label{1cap:spinta_laterale}
% [titolo ridotto se non ci dovesse stare] {titolo completo}
%

%\begin{citazione}
%BREVE SPIEGAZIONE CONTENUTO CAPITOLO
%\end{citazione}

\section{Prima sezione risultati}

\subsection{Prima sottosezione risultati}

\paragraph{Primo paragrafo risultati} Questa è una prova di un testo di un paragrafo.

The well known Pythagorean theorem $x^2 + y^2 = z^2$\footnote{\url{http://google.com}} was 
proved to be invalid \textbf{\textit{for other exponents}}. 

\textsc{com.java.xxx}

``Lorem ipsum''~\cite{di2017developer}

Meaning the next equation has no integer solutions:

\[ x^n_m + y^n = z^n \]

\begin{figure}[h]
\includegraphics[width=2cm]{immagini/picture_unisa.jpg}
\centering
\caption{Questo sono io}
\end{figure}

\begin{center}
\begin{table}[!h]
\begin{tabular}{||c c c ||}
 \hline
 Col1 & Col2 & Col3 \\ [0.5ex] 
 \hline\hline
 1 & 6 & 787 \\ 
 \hline
 2 & 7 & 5415 \\
 \hline
 3 & 545 & 7507 \\
 \hline
 4 & 545 & 7560 \\
 \hline
 5 & 88 & 6344 \\ [1ex] 
 \hline
\end{tabular}
\caption{Esempio di tabella}
\end{table}
\end{center}

The well known Pythagorean theorem $x^2 + y^2 = z^2$ was 
proved to be invalid for other exponents. 
Meaning the next equation has no integer solutions:\cite{bruegge2009object}

\[ x^n_m + y^n = z^n \]

\begin{minted}
[
frame=lines,
framesep=2mm,
baselinestretch=1.2,
bgcolor=LightGray,
fontsize=\footnotesize,
linenos
]
{python}
import numpy as np
    
def incmatrix(genl1,genl2):
    m = len(genl1)
    n = len(genl2)
    M = None #to become the incidence matrix
    VT = np.zeros((n*m,1), int)  #dummy variable
    
    #compute the bitwise xor matrix
    M1 = bitxormatrix(genl1)
    M2 = np.triu(bitxormatrix(genl2),1) 

    for i in range(m-1):
        for j in range(i+1, m):
            [r,c] = np.where(M2 == M1[i,j])
            for k in range(len(r)):
                VT[(i)*n + r[k]] = 1;
                VT[(i)*n + c[k]] = 1;
                VT[(j)*n + r[k]] = 1;
                VT[(j)*n + c[k]] = 1;
                
                if M is None:
                    M = np.copy(VT)
                else:
                    M = np.concatenate((M, VT), 1)
                
                VT = np.zeros((n*m,1), int)
    
    return M
\end{minted}

Questo è un elenco puntato
\begin{itemize}
    \item primo elemento
    \begin{itemize}
    \item primo sottoelemento
    \end{itemize}
    \item secondo elemento
    \item terzo elemento
\end{itemize}

Questo è un elenco numerato
\begin{enumerate}
    \item primo elemento
    \begin{enumerate}
    \item primo sottoelemento
    \end{enumerate}
    \item secondo elemento
    \item terzo elemento
\end{enumerate}

\newpage