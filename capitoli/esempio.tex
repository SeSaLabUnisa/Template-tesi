\chapter{Esempio Capitolo} %\label{1cap:spinta_laterale}
% [titolo ridotto se non ci dovesse stare] {titolo completo}
%

%\begin{citazione}
%BREVE SPIEGAZIONE CONTENUTO CAPITOLO
%\end{citazione}

\section{Prima sezione}

\subsection{Prima sottosezione}

\paragraph{Primo paragrafo} Questa è una prova di un testo di un paragrafo.

The well known Pythagorean theorem $x^2 + y^2 = z^2$\footnote{\url{http://google.com}} was 
proved to be invalid \textbf{\textit{for other exponents}}. 

\textsc{com.java.xxx}

``Lorem ipsum''

Meaning the next equation has no integer solutions:

\[ x^n_m + y^n = z^n \]

\begin{figure}[h]
\includegraphics[width=3cm]{figure/picture.pdf}
\centering
\caption{Girasole piantato nel cortile del Dipartimento di Informatica.}
\end{figure}






\begin{table}
    \centering
    \caption{Questions to decide whether to conduct a MLR~\cite{garousi2019_mlr_guidelines}.}
    \vspace{1mm} % Adjust the height of the space between caption and tabular
    
    \rowcolors{1}{graytable}{white}
    \resizebox{\linewidth}{!}{
    \begin{tabular}{p{0.025\linewidth} p{0.875\linewidth} P{0.1\linewidth}}
        \toprule
        \rowcolor{black}
        \textbf{\textcolor{white}{\#}} & \textbf{\textcolor{white}{Question}} & \textbf{\textcolor{white}{Answer}}\\
        \bottomrule
        1 & Is the subject ``complex'' and not solvable by considering only the formal literature? & Yes\\
        2 & Is there a lack of volume or quality of evidence, or a lack of consensus of outcome measurement in the formal literature? & No\\
        3 & Is the contextual information important to the subject under study? & Yes\\
        4 & Is it the goal to validate or corroborate scientific outcomes with practical experiences? & Yes\\
        5 & Is it the goal to challenge assumptions or falsify results from practice using academic research or vice versa? & Yes\\
        6 & Would a synthesis of insights and evidence from the industrial and academic community be useful to one or even both communities? & Yes\\
        7 & Is there a large volume of practitioner sources indicating high practitioner interest in a topic? & Yes\\
        \bottomrule
        \rowcolor{white}
        \multicolumn{3}{p{\linewidth}}{\textit{\textbf{Note}: The possible answers to each question are ``Yes'' or ``No''. One or more ``Yes'' responses suggest that it could be useful to conduct a MLR.}}\\
    \end{tabular}
    }
    
    \label{table:motivations_MLR}
\end{table}









\begin{center}
\begin{table}[!h]
\begin{tabular}{||c c c ||}
 \hline
 Col1 & Col2 & Col3 \\ [0.5ex] 
 \hline\hline
 1 & 6 & 787 \\ 
 \hline
 2 & 7 & 5415 \\
 \hline
 3 & 545 & 7507 \\
 \hline
 4 & 545 & 7560 \\
 \hline
 5 & 88 & 6344 \\ [1ex] 
 \hline
\end{tabular}
\caption{Esempio di tabella}
\end{table}
\end{center}

The well known Pythagorean theorem $x^2 + y^2 = z^2$ was 
proved to be invalid for other exponents. 
Meaning the next equation has no integer solutions:

\[ x^n_m + y^n = z^n \]

\begin{minted}
[
frame=lines,
framesep=2mm,
baselinestretch=1.2,
bgcolor=LightGray,
fontsize=\footnotesize,
linenos
]
{python}
import numpy as np
    
def incmatrix(genl1,genl2):
    m = len(genl1)
    n = len(genl2)
    M = None #to become the incidence matrix
    VT = np.zeros((n*m,1), int)  #dummy variable
    
    #compute the bitwise xor matrix
    M1 = bitxormatrix(genl1)
    M2 = np.triu(bitxormatrix(genl2),1) 

    for i in range(m-1):
        for j in range(i+1, m):
            [r,c] = np.where(M2 == M1[i,j])
            for k in range(len(r)):
                VT[(i)*n + r[k]] = 1;
                VT[(i)*n + c[k]] = 1;
                VT[(j)*n + r[k]] = 1;
                VT[(j)*n + c[k]] = 1;
                
                if M is None:
                    M = np.copy(VT)
                else:
                    M = np.concatenate((M, VT), 1)
                
                VT = np.zeros((n*m,1), int)
    
    return M
\end{minted}

Questo è un elenco puntato
\begin{itemize}
    \item primo elemento
    \begin{itemize}
    \item primo sottoelemento
    \end{itemize}
    \item secondo elemento
    \item terzo elemento
\end{itemize}

Questo è un elenco numerato
\begin{enumerate}
    \item primo elemento
    \begin{enumerate}
    \item primo sottoelemento
    \end{enumerate}
    \item secondo elemento
    \item terzo elemento
\end{enumerate}

Questa è una frase che dice qualcosa, e deve essere supportata da una citazione.
Se gli autori sono più di due, scriviamo che Garousi et al. hanno detto qualcosa~\cite{garousi2019_mlr_guidelines}.
Se gli autori sono due, scriviamo che Bruegge e Dutoit hanno detto cose~\cite{bruegge2009object_se}.
Quest'altra frase dice qualcos'altro, ma non so in quale paper l'ho letta~\needcite.
Questa frase dice cose supportate da diverse pubblicazioni~\cite{casillo2022detecting,destefano2020splicing,giordano2022slr,iannone2022secret,lambiase2022fences,sellitto2022refactoring,pontillo2022static}.
In questa parte devo ricordarmi qualcosa\nb{devo ricordarmi di questa cosa}.

\revised{Questa parte del documento è stata modificata dopo che il relatore mi ha dato suggerimenti.}

Invece qui ho avuto dei feedback dal relatore, e devo ricordarmi di applicare i commenti. \feedback{aggiungi un paragrafo sulle research question}

\goal{Questo è l'obiettivo della tesi.}

\rqbox{1}{Quale è la nostra prima Research Question?}

\rqanswer{1}{Questa è la risposta alla prima RQ.}

\finding{1}{Questo è il primo finding che riportiamo tra i risultati.}

\takeaway{1}{Primo takeaway message.}

