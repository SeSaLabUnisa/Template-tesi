\chapter{Introduzione}

\begin{comment}
Ciò che viene scritto in questo enviroment non sarà mostrato nel pdf.
Questo comando è utile per nascondere un intero paragrafo senza eliminarlo dal file sorgente. 
\end{comment}
Alcune linee guida per scrivere la tesi: 1) Ogni immagine/tabella deve avere le seguenti caratteristiche: a) Una caption; b) Deve essere esplicitamente commentata all'interno del testo (ricordatevi che se vi riferite alla figura/tabella inserendo anche il corrispettivo numero e.g., Figura X mostra che... allora la prima lettera deve essere scritta in maiuscolo; E ricordatevi che latex mette a disposizione un comando per citare correttamente (ref) senza dover mettere a mano il numeretto corrispondente) c) Le immagini devono essere rifatte tramite PPT o similari e salvate in pdf prima di inserirle nel file latex, altrimenti rischiate di inserire immagini sgranate. 2) Ricordatevi sempre di citare le fonti e di rifrasare, altrimenti rischiate di inccorrere nel reato di plagio (che può comportare anche la revoca del titolo accademico) e che citare non sminuisce il vostro lavoro, anzi, lo rafforza. 3)L'Abstract deve necessariamente contenere le seuguenti informazioni a) Un breve riasssunto della problematica; b)Una breve descrizione di ciò che è stato fatto; c) I risultati ottenuti.
